\section{Conclusion}
The results underlines the notion that using a prior model and using fixed parameter is of no difference. Evidently from the pictures in ..., the two methods also yield the same predictions.  However, the quality of simulations are somewhat poor if they are to be used in such a comparison as we don't have too much theory to aid us in separating the two procedures. The final conclusion based on the results is that the optimal paths are the combinations that cover the most of the field, but again, the simulations was possible performed with too few sample paths to compare.

With respect to the criterias presented in section (\ref{}) 4, the fixed parameter procedure would be preferred with the main argument being the difference in computational cost. As described, this yielded a quicker calculation with the same results and is therefore preferred. 

Recommende further work would be to possible redefine the model in such a way to exploit the sparseness of the covariancematrix. As the case is presented with a variogram, one could possible redefine the model as a Gaussian Random Markov Field. One could also utilize INLA as an approximation technique, as the latent variables are modelled by a Gaussian process in both cases.