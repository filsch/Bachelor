\section{Introduction}
%Skriv en innledning, hva handler oppgaven om (tema, problemstilling, hensikten:hva er det du vil finne ut), hva var motivasjonen for den

This report is written in coherence with the bachelor project for my studies within the Department of Mathematical Sciences at NTNU. The overall focus of this project is a case that will be handled with tools from spatial statistics. 

Spatial statistics refers to the application of statistical methods in a spatial setting. Statistics in itself is an extremely useful mathematical tool to quantify uncertainty. Uncertainty is something that we all face in a day to day basis in our daily lives, but perhaps even more so when viewing the world through scientific research. How can we know that our measurements, and thus how the world is described, are certain? In this sense, Cressie \& Wikle refers to statistics as the "Science of Uncertainty". Statistics has many ways of dealing with uncertainty, and one of them is that is interprets uncertainty as a measure of \textit{variability}. Other interpretations could also be used, like \textit{entropy}. By modelling uncertainty via variability, one may model the total variability by the variability in measurements, the model used and due to the uncertainty of parameters in the model \cite{CressieEtAl}. All this happens with the focus on acquiring information from the data that are to be modelled. In spatial statistics, all this happens in a spatial setting, i.e. the data that are to be analyzed via their topological, geometric or geographic attributes.

A typical model thats used in spatial statistics is a \textit{Gaussian Random Field} (GRF). Informally, a GRF is a collection of data that have been given a multivariate Gaussian joint probability function. The random field that it is 'attached' to typically defines the relation between the variables in some quantifiable way. A typical example (that will also be used extensively) is spatial location. As the joint probability function is multivariate Gaussian, all subsets of variables that are included in the GRF is also multivariate Gaussian with their attributes still defined by the relation inherited from the GRF. Due to the many useful properties of a multivariate Gaussian distributed variable, the GRF is a useful way to model data. 

In order to model data in a easy to interpret way, a model for GRF is often used to describe the various parts of the model. An example that wil be used in this project is the \textit{Bayes hierarchical model} (BHM). The exact formulation of a BHM is presented later, but summarized it consists of a 1) data model for specifying the data acquirement, 2) a process model for describing the latent process that is typically of interest and lastly any parameter-priors that are used in the Bayesian setting in 3) prior model. By distinguishing between the three, interpretability is made easy and one gets an intuitive way to understand the data acquirement how it relates to the latent process that is often of interest. 

As the latent process is of interest, making the correct data acquirement is crucial in order to achieve sufficient information from the data. In spatial statistics this is important, as the data must then be sampled at the correct locations. A set of measurement-locations is called a \textit{sampling design}, or simply \textit{design}, in spatial setting. Choosing the correct sampling design may be crucial for achieving the desired information, and must be evaluated with possible limitations as cost and precision of measurements. Constructing sampling designs happens typically in two settings, \textit{retroactive design} and \textit{proactive design}. A retrospective design will focus on altering the existing design on the basis of new data, while a proactive design will focus on constructing the optimal design on the basis of prior and estimated posterior data. 

This paper is organized into 5 sections, including the introduction and conclusion. Section 2 introduces the GRF by going through some basics and relevant properties of random fields, covariance functions and Gaussian processes. A short summary of \textit{Generalized Least Squares} (GLS) and hierarchical models is also part of this section, as it is to be used in the case scenario. The posterior distribution needed for the case scenario in section 4 is also derived. Section 3 introduces the notion of spatial sample design and how it may be related to the model in a quantifiable way. Some examples of path planning criterias is presented along with a short summary of the situation it were used in. Section 4 presents a case that the theory of section 2 and 3 will be applied upon. Input for the case and chosen modelspecification is presented, along with some case-specific procedures for estimating parameters based on the input data. Results follow in the last subsection of section 4, before the conclusion round up the main points of this paper.  

\subsection*{Notation} 
In this paper I'll denote matrices with a capital symbol. Vectors will be denoted by a arrow over its lower-case symbol. An exception will be in the case of a vector of stochastic variables, in that case, a large letter is used. Throughout the paper, only the symbols $A, B, Z, Z_a, Y$ is used for this. 
Lastly, a matrix $F(\vec{s})$ will be introduced later and used extensively. At some places the notation $Z(\vec{s}) / Y(\vec{s})$ will be used in place of $F(\vec{s}) \cdot Z$ / $F(\vec{s}) \cdot Y$.
To summarize, with $X$ a matrix,
\begin{equation*}
    X = 
    \begin{bmatrix}
    x_{11} & x_{12} & \dots & \\
    x_{21} & x_{22} & \dots & \\
    \vdots & \vdots & \ddots & \\
    & & & 
    \end{bmatrix}
\end{equation*}
$\vec{\beta}$ a vector
\begin{equation*}
    \vec{\beta} = \begin{bmatrix} \beta_0 & \beta_1 & \dots \end{bmatrix}
\end{equation*}
where $x_{ii}$ and $\beta_i$ are values $\forall$ i, and $Y$ as a vector of stochastic variables
\begin{equation*}
    Y = \begin{bmatrix} Y_0 & Y_1 & \dots \end{bmatrix}
\end{equation*}
where $Y_i$ is a stochastic variable $\forall$ i