\section{Spatial Design}
\subsection{Sampling design} 
\label{sec:sampling_design}
In spatial prediction, the choice of which spatial locations, or spatial design, to sample data from becomes critical. In the model setting, this means defining how our matrix $F(\vec{s})$ mentioned in (\ref{}) is constructed. In order to achieve the best design, one would have to choose in cohesion with the model that is defined for the data situation and as to what criteria the model is trying to satisfy. 

There are two common situations when producing sampling designs. Firstly we have the \textit{retrospective design}, where one assesses the sampling design after obtaining information about the attributes of interest. A goal in this setting might be to reduce the number of sampling locations without a significant loss in the amount or quality of information. This problem is exemplified by a want to economize measuring-stations while still achieving satisfactory measurements from the remaining locations.  

The second approach is that of \textit{proactive design}. In this form, the overall goal is to design a set of sampling locations in advance of the data acquisitioning that align with some criteria set for the project. The criteria is most often set on a applicationbased setting, indicating that it doesn't exist any general solution to produce the optimal design. Typically in a predictive setting, criterias that revolve around minimizing the variance or predicitve error of a design or maximizing the entropy is common. Two examples of criterias relating to the variance are found in equations (\ref{eq:criteria_binney}) and (\ref{eq:criteria_diggle}). 


The setting in which this criteria was used was the desire to find the optimal path through a graph that were modelled as a random field \cite{BinneyEtAl}. The criteria was to minimize the \textit{Expected Mean Squared Error} (EMSE), that they calculated as 
\begin{equation} \label{eq:criteria_binney}
\text{EMSE} = \frac{1}{n} [tr(\Sigma_p) - tr(\Sigma_o)]
\end{equation}
where $\Sigma_p$ denotes the posterior covariancematrix and $\Sigma_o$ denotes the prior covariancematrix.

For equation (\ref{eq:criteria_diggle}), the setting was retroactive design planning, were the ultimate goal was to remove as many measurement-locations in a random field without significant loss in measurement amount or quality \cite{DiggleEtAl}. This criteria was adhered to by minimizing the equation 
\begin{equation} \label{eq:criteria_diggle}
E(\vec{\nu}) = \int_A E_{Y|\theta_0} [Var(S(x)|Y)]dx
\end{equation}
where $S(x)$ denotes a function of the proposed reduced sampling design, $Y$ known data and $\theta_0$ the true parameter values.

To adhere to the criteria that's set, the design need not be solved within the workframe as stated by the model ones uses. Diggle \& Lophaven \cite{DiggleEtAl} refers to some purely geometrical approaches that may work well for finding adequate designs. In this project a satisfiable design will be estimated on the basis of a combination of predefined design proposals, i.e. the project will use focus on proactive design. The criteria that will be used in this project is minimizing the spatially averaged standard deviation of the proposed sample design, i.e. 
\begin{equation} \label{eq:criteria}
I = \sum_{Z_a} \sum_{i=1}^n \sqrt{ Var (Y_i | Z_a, Z) } P(Z_a | Z) .
\end{equation}
where $Z_a$ denotes the proposed design, $Z$ current data that has been acquired and $Y_i$ as the latent process at the location to be predicted. 

\subsection{Constructing designs in Bayesian hierarchical models}

With a Bayesian hierarchical model, one has to make some alterations to the calculation. In our setting, with prior distributions for $\sigma^2$ and $\tau$, we need to evaluate the integral that arises from 
\begin{equation*}
P(Z_a | Z) = \int \int P(Z_a | Z, \sigma^2, \tau) d\sigma^2 d\tau
\end{equation*}
In stead of evaluating this double-integral, Monte Carlo sampling of $P(Z_a | Z)$ is used to estimate $I$. To generate a sample $Z_a$, a realization of both $\sigma^2$ and $\tau$ is drawn from their respective distributions.
With these, a sample $Z_a^*$ is drawn from $P(Z_a | Z, \sigma^2, \tau)$, and evaluating the criteria as 
\begin{equation*}
I = \sum_{Z_a^*} \sum_{i=1}^n \sqrt{ Var (Y_i | Z_a^*, Z) }.
\end{equation*}
Performing this Monte Carlo sampling an adequate amount of times, we obtain a result for our criteria. 